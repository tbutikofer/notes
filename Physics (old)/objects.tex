\section{Quantified World}
\subsection{Objects and subjects}
Following the age old approach to reality of natural philosophy to distinguish between the observer and the observed world, I'd like to start this treatise with an attempt of an abstraction what it means to be an observer and what it means to be observed.\\
It is an intrinsic property of an abstraction to focus on a limited set of properties, such that it allows to build a model in an attempt to describe the observed reality by reducing it to the chosen abstraction. The following pages are no different.\\
In this sense I define an observer\footnote{Some schools of philosophies use the term \href{https://en.wikipedia.org/wiki/Subject_(philosophy)}{subject}.} as a set $\mathcal{O}$ of possible measurements (or senses). The observed object $\Phi$ is completely described by $\mathcal{O}$ by the result set of all measurements $\mathcal{O}(\Phi)$.\\
Note that the term $\mathcal{O}(\Phi)$ implies the assumption that an object $\Phi$ has an existence independent  of the observer $\mathcal{O}$. Different observers might perform different measurements  of the same object.
\begin{axiom}
Objects exist independent of any observer.
\end{axiom}
 try how far this reduction 

The distinction and interaction between the observable universe and the observer is a traditional topic for philosophical discourses. Physics as a natural philosophical discipline makes no exception. We start this overview of my understanding of the mathematical description of the world's behaviour therefore with the definition of a physical object:\\
A physical object is identified with a complete description of the physical object.
The description of a physical object is complete if it contains the results of all possible measurements of the physical object in question.\\
Since the topic of this treatise is an overview of the physical world, the term "object" are always to be understood as "physical object".\\
This definition provokes the question: What is a measurement?
 

A subject is an observer and an object is a thing observed. An object is a set of properties, i.e. measurements. A property consists of a measurement on an object and its result. An red apple is an object for which the measurement of its color results in red.
\subsection{Measurements}
Measurements enable an observer to relate different objects by comparison. A property of an object is only meaningful if it can be compared to the same property of another object. E.g.: The property ”color” is only meaningful if there exist other objects with different values of this property. Saying an apple has the property ”rupalsy” of (let’s say) 5 is meaningless unless we find other objects on which the measurement of ”rupalsy” reveals this specific property of the objects in question.\\
Length, mass, time intervals, electrical charge are examples of measurements. But also volume, temperature, torque etc.\\
Since an object is defined by its properties, each the result of a measurement, an object can be represented as an element of the space of all measurement results: the state space of the object.
Calling an object’s properties the state of an object implies a possible transition from one set of properties to another with an equivalency of these two states by attributing them to the same object. I.e. even if an oven changes its tempera- ture, its still the same oven.
